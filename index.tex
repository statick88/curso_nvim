% Options for packages loaded elsewhere
\PassOptionsToPackage{unicode}{hyperref}
\PassOptionsToPackage{hyphens}{url}
\PassOptionsToPackage{dvipsnames,svgnames,x11names}{xcolor}
%
\documentclass[
  a4paper,
  DIV=11,
  numbers=noendperiod,
  onepage,
  openany]{scrreprt}

\usepackage{amsmath,amssymb}
\usepackage{iftex}
\ifPDFTeX
  \usepackage[T1]{fontenc}
  \usepackage[utf8]{inputenc}
  \usepackage{textcomp} % provide euro and other symbols
\else % if luatex or xetex
  \usepackage{unicode-math}
  \defaultfontfeatures{Scale=MatchLowercase}
  \defaultfontfeatures[\rmfamily]{Ligatures=TeX,Scale=1}
\fi
\usepackage{lmodern}
\ifPDFTeX\else  
    % xetex/luatex font selection
\fi
% Use upquote if available, for straight quotes in verbatim environments
\IfFileExists{upquote.sty}{\usepackage{upquote}}{}
\IfFileExists{microtype.sty}{% use microtype if available
  \usepackage[]{microtype}
  \UseMicrotypeSet[protrusion]{basicmath} % disable protrusion for tt fonts
}{}
\makeatletter
\@ifundefined{KOMAClassName}{% if non-KOMA class
  \IfFileExists{parskip.sty}{%
    \usepackage{parskip}
  }{% else
    \setlength{\parindent}{0pt}
    \setlength{\parskip}{6pt plus 2pt minus 1pt}}
}{% if KOMA class
  \KOMAoptions{parskip=half}}
\makeatother
\usepackage{xcolor}
\usepackage[lmargin=30mm,rmargin=30mm,tmargin=35mm,bmargin=30mm]{geometry}
\setlength{\emergencystretch}{3em} % prevent overfull lines
\setcounter{secnumdepth}{5}
% Make \paragraph and \subparagraph free-standing
\ifx\paragraph\undefined\else
  \let\oldparagraph\paragraph
  \renewcommand{\paragraph}[1]{\oldparagraph{#1}\mbox{}}
\fi
\ifx\subparagraph\undefined\else
  \let\oldsubparagraph\subparagraph
  \renewcommand{\subparagraph}[1]{\oldsubparagraph{#1}\mbox{}}
\fi


\providecommand{\tightlist}{%
  \setlength{\itemsep}{0pt}\setlength{\parskip}{0pt}}\usepackage{longtable,booktabs,array}
\usepackage{calc} % for calculating minipage widths
% Correct order of tables after \paragraph or \subparagraph
\usepackage{etoolbox}
\makeatletter
\patchcmd\longtable{\par}{\if@noskipsec\mbox{}\fi\par}{}{}
\makeatother
% Allow footnotes in longtable head/foot
\IfFileExists{footnotehyper.sty}{\usepackage{footnotehyper}}{\usepackage{footnote}}
\makesavenoteenv{longtable}
\usepackage{graphicx}
\makeatletter
\def\maxwidth{\ifdim\Gin@nat@width>\linewidth\linewidth\else\Gin@nat@width\fi}
\def\maxheight{\ifdim\Gin@nat@height>\textheight\textheight\else\Gin@nat@height\fi}
\makeatother
% Scale images if necessary, so that they will not overflow the page
% margins by default, and it is still possible to overwrite the defaults
% using explicit options in \includegraphics[width, height, ...]{}
\setkeys{Gin}{width=\maxwidth,height=\maxheight,keepaspectratio}
% Set default figure placement to htbp
\makeatletter
\def\fps@figure{htbp}
\makeatother

\KOMAoption{captions}{tableheading}
\makeatletter
\@ifpackageloaded{caption}{}{\usepackage{caption}}
\AtBeginDocument{%
\ifdefined\contentsname
  \renewcommand*\contentsname{Table of contents}
\else
  \newcommand\contentsname{Table of contents}
\fi
\ifdefined\listfigurename
  \renewcommand*\listfigurename{List of Figures}
\else
  \newcommand\listfigurename{List of Figures}
\fi
\ifdefined\listtablename
  \renewcommand*\listtablename{List of Tables}
\else
  \newcommand\listtablename{List of Tables}
\fi
\ifdefined\figurename
  \renewcommand*\figurename{Figure}
\else
  \newcommand\figurename{Figure}
\fi
\ifdefined\tablename
  \renewcommand*\tablename{Table}
\else
  \newcommand\tablename{Table}
\fi
}
\@ifpackageloaded{float}{}{\usepackage{float}}
\floatstyle{ruled}
\@ifundefined{c@chapter}{\newfloat{codelisting}{h}{lop}}{\newfloat{codelisting}{h}{lop}[chapter]}
\floatname{codelisting}{Listing}
\newcommand*\listoflistings{\listof{codelisting}{List of Listings}}
\makeatother
\makeatletter
\makeatother
\makeatletter
\@ifpackageloaded{caption}{}{\usepackage{caption}}
\@ifpackageloaded{subcaption}{}{\usepackage{subcaption}}
\makeatother
\ifLuaTeX
  \usepackage{selnolig}  % disable illegal ligatures
\fi
\usepackage{bookmark}

\IfFileExists{xurl.sty}{\usepackage{xurl}}{} % add URL line breaks if available
\urlstyle{same} % disable monospaced font for URLs
\hypersetup{
  pdftitle={Curso Completo de Neovim: De Novato a Maestro},
  pdfauthor={Diego Saavedra},
  colorlinks=true,
  linkcolor={blue},
  filecolor={Maroon},
  citecolor={Blue},
  urlcolor={Blue},
  pdfcreator={LaTeX via pandoc}}

\title{Curso Completo de Neovim: De Novato a Maestro}
\author{Diego Saavedra}
\date{Nov 5, 2024}

\begin{document}
\maketitle

\renewcommand*\contentsname{Table of contents}
{
\hypersetup{linkcolor=}
\setcounter{tocdepth}{2}
\tableofcontents
}
\chapter{Curso Completo de Neovim: De Novato a
Maestro}\label{curso-completo-de-neovim-de-novato-a-maestro}

\begin{figure}[H]

{\centering \includegraphics[width=2.08333in,height=\textheight]{../../../logo.png}

}

\caption{Neovim}

\end{figure}%

\section{Descripción del Curso}\label{descripciuxf3n-del-curso}

Este curso tiene como objetivo llevar al estudiante desde un nivel
básico de comprensión de Neovim hasta un nivel avanzado de
personalización, edición eficiente y automatización del flujo de
trabajo. Usaremos Lazy como gestor de plugins, junto con herramientas
como LazyGit y Gitsigns para optimizar el trabajo en Git, y exploraremos
cómo integrar Neovim con herramientas modernas de desarrollo.

\section{Planificación del Curso}\label{planificaciuxf3n-del-curso}

\subsection{\texorpdfstring{\textbf{Módulo 1}: Introducción a
Neovim}{Módulo 1: Introducción a Neovim}}\label{muxf3dulo-1-introducciuxf3n-a-neovim}

\textbf{Objetivo}: Familiarizarse con Neovim y sentar las bases de su
configuración y personalización.

\textbf{Lección 1}: Introducción al Curso y Objetivos

\begin{itemize}
\item
  Historia de Vim y Neovim.
\item
  Beneficios y características de Neovim.
\item
  Estructura del curso y recursos adicionales.
\end{itemize}

\textbf{Lección 2}: Instalación de Neovim y Configuración Básica

\begin{itemize}
\item
  Instalación en Windows y Linux.
\item
  Creación de la carpeta de configuración y archivos principales
  (init.lua o init.vim).
\item
  Instalación de Lazy como gestor de plugins.
\end{itemize}

\subsection{\texorpdfstring{\textbf{Módulo 2}: Fundamentos de Neovim y
Comandos
Básicos}{Módulo 2: Fundamentos de Neovim y Comandos Básicos}}\label{muxf3dulo-2-fundamentos-de-neovim-y-comandos-buxe1sicos}

\textbf{Objetivo}: Aprender los comandos esenciales de Neovim y la
navegación básica.

\textbf{Lección 1}: Navegación y Modos de Neovim

\begin{itemize}
\item
  \textbf{Modos de operación:} Normal, Insert, Visual, Command.
\item
  Navegación en el archivo y comandos básicos de movimiento.
\end{itemize}

\textbf{Lección 2}: Edición de Texto y Búsqueda en Neovim

\begin{itemize}
\item
  Comandos para edición, eliminación, y copia de texto.
\item
  Búsqueda y reemplazo de texto en Neovim.
\end{itemize}

\textbf{Lección 3}: Configuración Inicial de Opciones

\begin{itemize}
\item
  Configuración de indentación, color de fondo y fuentes.
\item
  Introducción al archivo init.lua y a las opciones de personalización.
\end{itemize}

\subsection{\texorpdfstring{\textbf{Módulo 3}: Configuración y
Personalización de Plugins con
Lazy}{Módulo 3: Configuración y Personalización de Plugins con Lazy}}\label{muxf3dulo-3-configuraciuxf3n-y-personalizaciuxf3n-de-plugins-con-lazy}

\textbf{Objetivo}: Instalar y configurar plugins clave que aumentarán la
eficiencia en Neovim.

\textbf{Lección 1}: Gestión de Plugins con Lazy

\begin{itemize}
\item
  Instalación y configuración de Lazy.
\item
  Estrategias para organizar plugins en Lazy.
\end{itemize}

\textbf{Lección 2}: Exploración de Plugins de Edición y Productividad

\begin{itemize}
\item
  Instalación de plugins como nvim-treesitter para resaltado avanzado de
  sintaxis y comment.nvim para gestión de comentarios.
\item
  Configuración y personalización de cada plugin para adaptarlos al
  flujo de trabajo.
\end{itemize}

\textbf{Lección 3}: Integración con Git usando LazyGit y Gitsigns

\begin{itemize}
\item
  Instalación de LazyGit y configuración de accesos directos para su uso
  en Neovim.
\item
  Uso de Gitsigns para la visualización de cambios en el código.
\item
  Práctica guiada de flujos de trabajo comunes con Git.
\end{itemize}

\subsection{\texorpdfstring{\textbf{Módulo 4}: Dominio de Edición y
Navegación Avanzada en
Neovim}{Módulo 4: Dominio de Edición y Navegación Avanzada en Neovim}}\label{muxf3dulo-4-dominio-de-ediciuxf3n-y-navegaciuxf3n-avanzada-en-neovim}

\textbf{Objetivo}: Profundizar en técnicas de edición avanzada y
optimizar la navegación.

\textbf{Lección 1}: Movimientos Avanzados y Edición con Macros

\begin{itemize}
\item
  Navegación rápida por archivos grandes y uso de marcadores.
\item
  Creación y uso de macros para automatizar tareas repetitivas.
\end{itemize}

\textbf{Lección 2}: Uso de Bufers, Ventanas y Pestañas

\begin{itemize}
\item
  Explicación de buffers, ventanas y pestañas.
\item
  Configuración para alternar y organizar múltiples archivos
  eficientemente.
\end{itemize}

\textbf{Lección 3}: Configuración de Snippets con LuaSnip

\begin{itemize}
\item
  Instalación de LuaSnip y configuración de snippets personalizados.
\item
  Creación de snippets de código para lenguajes específicos.
\end{itemize}

\subsection{\texorpdfstring{\textbf{Módulo 5}: Configuración Avanzada y
Personalización en
Neovim}{Módulo 5: Configuración Avanzada y Personalización en Neovim}}\label{muxf3dulo-5-configuraciuxf3n-avanzada-y-personalizaciuxf3n-en-neovim}

\textbf{Objetivo}: Explorar configuraciones avanzadas y aprender a
personalizar Neovim a un alto nivel.

\textbf{Lección 1}: Personalización de Atajos de Teclado y Mapeos

\begin{itemize}
\item
  Cómo crear mapeos personalizados en init.lua.
\item
  Integración de mapeos para plugins instalados.
\end{itemize}

\textbf{Lección 2}: Configuración de LSP (Language Server Protocol)

\begin{itemize}
\item
  Explicación del concepto de LSP y su utilidad en la autocompletación y
  análisis de código.
\item
  Instalación y configuración del cliente de LSP en Neovim para varios
  lenguajes.
\end{itemize}

\textbf{Lección 3}: Exploración de nvim-cmp y telescope.nvim

\begin{itemize}
\item
  Configuración de nvim-cmp para autocompletado inteligente.
\item
  Uso de telescope.nvim para buscar archivos, símbolos y contenido en el
  proyecto.
\end{itemize}

\subsection{\texorpdfstring{\textbf{Módulo 6}: Integración y
Automatización del Flujo de
Trabajo}{Módulo 6: Integración y Automatización del Flujo de Trabajo}}\label{muxf3dulo-6-integraciuxf3n-y-automatizaciuxf3n-del-flujo-de-trabajo}

\textbf{Objetivo}: Aprender a integrar Neovim con otras herramientas y
automatizar tareas para mejorar la productividad.

\textbf{Lección 1}: Integración con Terminal Embebido y Comandos
Personalizados

\begin{itemize}
\item
  Cómo usar el terminal integrado en Neovim.
\item
  Ejemplos de comandos personalizados para tareas frecuentes.
\end{itemize}

\textbf{Lección 2}: Integración con Herramientas Externas y Workflows

\begin{itemize}
\item
  Integración de Neovim con herramientas como Git, Docker, y linters.
\item
  Ejemplos prácticos de automatización en flujos de desarrollo.
\end{itemize}

\textbf{Lección 3}: Configuración de Tareas y Scripts Automatizados

\begin{itemize}
\item
  Configuración de scripts en Neovim para realizar tareas comunes.
\item
  Ejemplos de tareas automáticas al guardar, abrir o cerrar archivos.
\end{itemize}

\subsection{\texorpdfstring{\textbf{Módulo 7}: Proyecto Final y
Optimización de
Configuración}{Módulo 7: Proyecto Final y Optimización de Configuración}}\label{muxf3dulo-7-proyecto-final-y-optimizaciuxf3n-de-configuraciuxf3n}

\textbf{Objetivo}: Consolidar los conocimientos adquiridos en un
proyecto práctico y optimizar la configuración final de Neovim.

\textbf{Lección 1}: Creación de una Configuración Modular de Neovim

\begin{itemize}
\item
  Organización de los archivos de configuración en módulos.
\item
  Mejores prácticas para mantener una configuración escalable y
  organizada.
\end{itemize}

\textbf{Lección 2}: Proyecto Final - Entorno de Desarrollo Completo en
Neovim

\begin{itemize}
\item
  Los estudiantes crearán su propio entorno de desarrollo completo en
  Neovim.
\item
  Integración de todas las herramientas y plugins utilizados en el
  curso.
\item
  Personalización final para optimizar el flujo de trabajo.
\end{itemize}

\textbf{Lección 3}: Exportación y Respaldo de la Configuración

\begin{itemize}
\item
  Cómo versionar la configuración en GitHub.
\item
  Compartir configuraciones y mantener un respaldo en caso de migración.
\end{itemize}

\section{Materiales de Apoyo}\label{materiales-de-apoyo}

\begin{itemize}
\item
  \textbf{Repositorio en GitHub}: Ejemplos de configuración, ejercicios
  y soluciones.
\item
  \textbf{Documentación Complementaria}: Referencias a la documentación
  oficial de Neovim, Lazy, LSP, y otros plugins.
\item
  \textbf{Guías de Instalación}: Instrucciones específicas para instalar
  LazyGit, Gitsigns y otros complementos en diferentes sistemas
  operativos.
\item
  \textbf{Ejercicios Prácticos y Desafíos}: Actividades diseñadas para
  aplicar lo aprendido en cada módulo.
\end{itemize}

\section{Tareas Adicionales y
Proyectos}\label{tareas-adicionales-y-proyectos}

\begin{itemize}
\item
  Prácticas recomendadas al final de cada módulo para afianzar los
  conocimientos.
\item
  Proyectos opcionales como configurar Neovim para proyectos específicos
  (por ejemplo, proyectos en Python, JavaScript, etc.).
\end{itemize}

\section{Soporte y Comunidad}\label{soporte-y-comunidad}

\begin{itemize}
\item
  \textbf{Acceso a foros y grupo de Discord}: Para resolver dudas,
  compartir configuraciones y obtener feedback.
\item
  \textbf{Recursos adicionales}: Artículos, guías y tutoriales que
  complementan el curso.
\end{itemize}

\part{Unidad 0: Introducción a Neovim}

\chapter{Introduction a Neovim}\label{introduction-a-neovim}

Nvim es una herramienta que te permite editar texto de forma eficiente.
Es un editor de texto que se puede personalizar para adaptarse a tus
necesidades. A diferencia de otros editores de texto, Nvim se puede
utilizar completamente desde el teclado, lo que lo hace muy eficiente
para los programadores y escritores.



\end{document}
